\chapter{Programming Principles}

\section{SOLID}

\subsection{Single-Responsibility-Prinzip}

Das Single-Responsibility-Prinzip besagt, dass Klassen und Funktionen stets nur einen Grund zur Änderung haben sollen. Bei Morik gilt dies für alle Klassen. Ein einfaches Beipiel ist die \textit{EntryFactory}, welche Entries erstellt und nur verändert werden müsste, um dies anders zu erreichen. Das Prinzip ist ebenfalls erfüllt vom \textit{Cipher} Interface, welches sowohl ver- als auch entschlüsseln kann. Dies sind zwar unterschiedliche Operationen, jedoch ergibt es keinen Sinn die eine zu ändern, ohne die andere gleichermaßen anzupassen. Der einzige Grund die Klasse zu verändern, ist um die Verschlüsselung zu ändern.

\subsection{Open-Closed-Prinzip}

Das Open-Closed-Prinzip besagt, dass Klassen, die bereits korrekt funktionieren, erweitert werden können (durch Einführung neuer abgeleiteter Klassen). 
Die Interfaces in Morik (\textit{Cipher}, \textit{DatabaseInterface}, etc.) setzen die Einführung von Implementierungen voraus um verwendet zu werden und können folglich einfach um weitere Subklassen erweitert werden. Auch der Großteil der anderen Klassen könnte prinzipiell durch Verwerbung erweitert werden, auch wenn es dazu oft keinen Grund gibt. Bei Klassen, für die wir explizit vermeiden wollen, dass von ihnen geerbt wird gibt es das \textbf{final} Schlüsselwort in C++. Dies verwenden wir \zB für das ValueObject \textit{EntryID}. Dadurch wird aktiv vermieden, dass Funktionalität hinzugefügt wird, da ValueObjects keine Funktionalität haben sollen. Streng genommen, verletzt dies das Open-Closed Prinzip.

Trotzdem gilt das Prinzip für den Großteil der Code-Basis.

\subsection{Liskov-Substitution-Prinzip}

Es gibt mehrere Stellen in unserer Anwendung, an denen Liksov-Substitution demonstriert werden kann. Das Prinzip ist überall erfüllt, da wir stets mit den Interfaces arbeiten, ohne downcasting anzuwenden.

Ein gutes Beispiel um dies zu demonstrieren ist das \textit{Cipher} interface.

\begin{lstlisting}[language=C++]
class Cipher {
public:
    virtual std::string encrypt(std::string plain, std::string master) const = 0;

    virtual std::string decrypt(std::string cipher, std::string master) const = 0;
};
\end{lstlisting}

Die Domain Services, die dieses Interface verwenden um Verschlüsselung durchzuführen akzeptieren einen Pointer auf ein \textit{Cipher} Objekt als Konstruktur Argument. Es hat keinen Einfluss auf deren Funktionalität ob das übergebene Objekt vom Typ \textit{CBC\_Cipher<AES>}, \textit{CBC\_Cipher<Serpent>} oder eine beliebige andere Implementierung des Interfaces ist.

Es ist zwar wichtig, dass zur Verschlüsselung der gleiche Algorithmus verwendet wird, wie zur Entschlüsselung. Dies ist jedoch irrelevant im Bezug auf dieses Prinzip, denn egal mit welchem Subtype man \textit{Cipher} ersetzt, die Funktionalität ist unverändert: Ein String kann unter Angabe eines Schlüssels verschlüsselt oder entschlüsselt werden. Solange der Subtyp das Interface korrekt implementiert, sind Substitutionen möglich.
Problematische Verwendungen können natürlich zu Problemen führen, beispielsweise die Verwendung verschiedener \textit{Cipher} Instanzen für Ver- und Entschlüsselung würde bedeuten, dass die Instanzen nur paarweise ersetzt werden dürfen, da sonst falsche Ergebnisse entstehen würden.

\subsection{Dependency-Inversion-Prinzip}

Dieses Prinzip besagt, dass high-level Komponenten keine Abhängigkeiten auf lower-level Komponenten haben sollen. Ein klassisches Beispiel, ist die Abhängigkeit von Plugins. Die Klasse in der Applikationsschicht, die die Funktionalität, die das Plugin bereitstellt verwendet, soll nicht von der Plugin-spezifischen Implementierung abhängig sein. Eine solche Abhängigkeit würde die Funktionalität der Applikation abhängig von der Funktionalität des Plugins machen. Durch Definition eines Interfaces in der Applikationsschicht, das in der Pluginschicht implementiert werden muss, kann dies verhindert werden. Die tatsächlichen Instanzen der Plugin-Klassen werden durch Dependency Injection übergeben. In unserem Fall gibt es beispielsweise die Schnittstelle zur Datenbank, in der Dependency Inversion demonstiert wird. Die abstrakte Klasse \textit{AbstractSqlDatabase} in der Adapterschicht, wird von der konkreten Implementierung einer SQL-Datenbank (der Klasse \textit{SQLiteDatabase}) implementiert.

\section{DRY}

Morik ist frei von Redundanzen, bei denen tatsächliche Funktionalität mehrmals implementiert wird. Natürlich gibt es einige Methodenhierarchien, die leicht redundant sind (bspw. \textit{PasswordEncryptor::encrypt} -> \textit{Cipher::encrypt} -> \textit{CBC\_Cipher::encrypt}), jedoch sind diese notwendig um den Architektur-Anforderungen zu entsprechen.

Ein Beispiel für die geziehlte Vermeidung von Redundanz ist der Einatz von Templates der \textit{CBC\_Cipher} Klasse, um verschiedene Blockchiffren verwenden zu können, ohne mehrere Klassen von Hand zu implementieren\footnote{siehe \url{https://github.com/moorts/Morik/blob/latex\_principles/src/plugins/encryption/CBC\_Cipher.cpp}}.

\section{GRASP}

Im folgenden wird erläutert, inwiefern Morik die Prinzipien Low Coupling und High Cohesion des GRASP Models' erfüllt.

\subsection{Low Coupling}

Morik verwendet Interfaces und damit polymorphe Methodenaufrufe an vielen Stellen\footnote{bspw. PasswordEncryptor ruft Methoden des Cipher Interface auf, EntryRepository ruft Methoden des DatabaseInterface auf}. Damit sind die einzelnen Komponenten nur schwach an die Implementierungen dieser Interfaces gekoppelt.

Auch gibt es generell kaum statische Methodenaufrufe, wobei zumindest im Verschlüsselungs-Plugin die Schlüsselherleitung als statische Methode realisiert ist\footnote{siehe \url{https://github.com/moorts/Morik/blob/latex_principles/src/plugins/encryption/KeyDerivation.h}}. Dadurch ist \textit{CBC\_Cipher} abhängig von dieser Methode. Dies ist jedoch die einzige Klasse, die an diese statische Methode gekoppelt ist.

\subsection{High Cohesion}

Die Klassen in Morik sind alle sehr übersichtlich. Klassen wie der Domain Service \textit{PasswordEncryptor} machen genau das, was der Name impliziert (in diesem Fall Passwörter verschlüsseln). Wir haben darauf geachtet, dies bei allen Klassen zu erreichen. Das sorgt für hohe Kohäsion, da keine Attribute und Methoden für Dinge notwenig sind, die nicht direkt zur Funktionalität gehören. Manche Klassen, wie das \textit{DatabaseInterface}\footnote{siehe \url{https://github.com/moorts/Morik/blob/latex_principles/src/adapters/database/DatabaseInterface.h}}, haben zwar eine leicht höhere Anzahl Methoden, jedoch sind diese logisch kohärent: Das Einfügen, Entfernen und Bearbeiten von Einträgen sind Konzepte die klar zusammen gehören.
