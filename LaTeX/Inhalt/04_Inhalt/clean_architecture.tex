\chapter{Clean Architecture}
\section{Plugins}
\subsection{Datenbank}
Die Datenbank stellt eines der Plugins dar. Als Technologie wurde hier eine SQLite3 Datenbank verwendet. Die konkrete Implementierung der Schnittstelle zur SQLite-Datenbank befindet sich in der Klasse \href{https://github.com/moorts/Morik/blob/main/src/plugins/database/SQLiteDatabase.h}{\textit{SQLiteDatabase}}. Diese Klasse erbt von der abstrakten Klasse \href{https://github.com/moorts/Morik/blob/main/src/application/AbstractDatabase.h}{\textit{AbstractDatabase}}, die sich in der Applikationsschicht befindet. Hierdurch wird eine Dependency Inversion erreicht, da nun eine \textit{AbstractDatabase} von anderen Klassen verwendet werden kann, um SQL-Befehle auf der Datenbank auszuführen, statt eine konkrete \textit{SQLiteDatabase} zu verwenden, was die Abhängigkeit von innen nach außen laufen lassen würde. Der Klasse \href{https://github.com/moorts/Morik/blob/main/src/application/DbInterface.h}{\textit{DbInterface}} wird im Konstruktor eine solche \textit{AbstractDatabase} übergeben, was die Dependency Injection umsetzt. Des Weiteren implementiert die Klasse \textit{DbInterface} die eigentliche Funktionalität in Form der SQL-Anweisungen und führt diese lediglich über die konkrete Implementierung der \textit{AbstractDatabase} in der Datenbank aus. Somit geht keine Funktionalität verloren, wenn das Plugin durch eine andere Datenbanktechnologie ausgetauscht wird. Die Umsetzung der Funktionalität als SQL-Anweisungen bedeutet jedoch, dass ein Austausch des Plugins nur ohne Weiteres möglich ist, wenn die neue Datenbank ebenfalls eine SQL-Datenbank ist. Handelt es sich bei der neuen Datenbank jedoch um eine noSQL-Datenbank, so muss die Funktionalität, also die Abfragen, angepasst werden.\\
Auf die Benutzung von Prepared Statements wurde hier verzichtet, da die Datenbank lokal ist und nur der Benutzer Befehle auf ihr ausführt. Würde die Datenbank über eine öffentliche Schnittstelle angesteuert werden, so wäre dies nicht zu vernachlässigen.
