% ------------------------------------------------------------
% LaTeX Template für die DHBW zum Schnellstart!
% Original: https://github.wdf.sap.corp/vtgermany/LaTeX-Template-DHBW
% ------------------------------------------------------------
% ---- Präambel mit Angaben zum Dokument
\input{Inhalt/00_Latex/praeambel}

% ---- Elektronische Version oder Gedruckte Version?
% ---- Unterschied: Die elektronische Version enthält keinen Platzhalter für die Unterschrift
\usepackage{ifthen}
\newboolean{e-Abgabe}
\setboolean{e-Abgabe}{false}    % false=gedruckte Fassung

% ---- Persönlichen Daten:
\newcommand{\titel}{Morik Password Manager}
\newcommand{\titelheader}{Morik}
\newcommand{\arbeit}{Programmentwurf}
\newcommand{\studiengang}{Informatik}
\newcommand{\studienjahr}{2019}
\newcommand{\autorOne}{Moritz Gutfleisch}
\newcommand{\autorTwo}{Erik Zimmermann}
\newcommand{\autorReverse}{Nachname, Vorname}
\newcommand{\verfassungsort}{Karlsruhe}
\newcommand{\matrikelnrOne}{0000000}
\newcommand{\matrikelnrTwo}{0000000}
\newcommand{\kurs}{TINF19B1}
\newcommand{\bearbeitungsmonat}{Januar 2018}
\newcommand{\abgabe}{01. Februar 2018}
\newcommand{\bearbeitungszeitraum}{01.10.2017 - 31.01.2018}
\newcommand{\betreuerDhbw}{Daniel Lindner}

\input{Inhalt/00_Latex/kopfundFusszeile}

% ---- Hilfreiches
\newcommand{\zB}{z.\,B. }   % "z.B." mit kleinem Leeraum dazwischen (ohne wäre nicht korrekt)
\newcommand{\dash}{d.\,h. }

\newcommand{\code}[1]{\texttt{#1}} % Ist einfacher zu schreiben als ständig \texttt und erlaubt
                                   % Änderungen im Nachhinein, wenn man z.B. Inline-Code anders stylen möchte.

% ---- Silbentrennung (falls LaTeX defaults falsch / nicht gewünscht sind)
\hyphenation{HANA}         % anstatt HA-NA
\hyphenation{Graph-Script} % anstatt GraphS-cript

% ---- Beginn des Dokuments
\begin{document}
\setlength{\parindent}{0pt}              % Keine Paragraphen Einrückung.
                                         % Dafür haben wir den Abstand zwischen den Paragraphen.
\setcounter{secnumdepth}{2}              % Nummerierungstiefe fürs Inhaltsverzeichnis
\setcounter{tocdepth}{1}                 % Tiefe des Inhaltsverzeichnisses. Ggf. so anpassen,
                                         % dass das Verzeichnis auf eine Seite passt.
\sffamily                                % Serifenlose Schrift verwenden.

% ---- Vorspann
% ------ Titelseite
\singlespacing
\thispagestyle{empty}
\begin{titlepage}
\enlargethispage{4cm}

\vspace*{0.1cm}

\begin{center}
	\Huge{\textbf{\titel}}\\[1.5cm]
	\LARGE{\textbf{\arbeit}}\\[1cm]
	\normalsize{von}\\[1ex] \Large{\textbf{\autorOne}} \\[1ex]
	\normalsize{und}\\[1ex] \Large{\textbf{\autorTwo}} \\[1cm]
\end{center}

\begin{center}
	\vfill
	\begin{tabular}{ll}
		Abgabedatum:                     & \abgabe \\[0.2cm]
		Bearbeitungszeitraum:            & \bearbeitungszeitraum \\[0.2cm]
		Matrikelnummer, Student:            & \matrikelnrOne{}, \autorOne \\[0.2cm]
		Matrikelnummer, Student:            & \matrikelnrTwo{}, \autorTwo \\[0.2cm]
		Kurs:            & \kurs \\[0.2cm]
		Gutachter der Dualen Hochschule: & \betreuerDhbw \\[2cm]
	\end{tabular} 
\end{center}
\end{titlepage}
  % Titelseite
\newcounter{savepage}
\pagenumbering{Roman}                    % Römische Seitenzahlen
\onehalfspacing

% ------ Erklärung, Sperrvermerk, Abstact
\include{Inhalt/01_Standard/erklaerung}
\include{Inhalt/02_Abstract/abstract-en}
\include{Inhalt/02_Abstract/abstract-de}

% ------ Inhaltsverzeichnis
\singlespacing
\tableofcontents

% ------ Verzeichnisse
\renewcommand*{\chapterpagestyle}{plain}
\pagestyle{plain}
\chapter*{Abkürzungsverzeichnis}
\addcontentsline{toc}{chapter}{Abkürzungsverzeichnis} % Hinzufügen zum Inhaltsverzeichnis 

\begin{acronym}[SQLXX] % längstes Kürzel wird verw. für den Abstand zw. Kürzel u. Text

	% Alphabetisch selbst sortieren - nicht verwendete Kürzel rausnehmen!
	\acro{SQL}{Structured Query Language}

\end{acronym}
\listoffigures                          % Erzeugen des Abbildungsverzeichnisses 
\listoftables                           % Erzeugen des Tabellenverzeichnisses
\renewcommand{\lstlistlistingname}{Quellcodeverzeichnis}
\lstlistoflistings                      % Erzeugen des Listenverzeichnisses
\setcounter{savepage}{\value{page}}


% ---- Inhalt der Arbeit
\cleardoublepage
\pagenumbering{arabic}                  % Arabische Seitenzahlen für den Hauptteil
\setlength{\parskip}{0.5\baselineskip}  % Abstand zwischen Absätzen
\rmfamily
\renewcommand*{\chapterpagestyle}{scrheadings}
\pagestyle{scrheadings}
\onehalfspacing
\chapter{Einleitung}
Morik ist ein Passwort Manager, der die gespeicherten Passwörter verschlüsselt in einer Datenbank aufbewahrt. Zusätzlich zu den Passwörtern wird ein Name des Eintrags gespeichert sowie ein optionaler Login. Neue Einträge können sehr einfach hinzugefügt werden. Von jedem vorhandenen Eintrag kann der Benutzer das Passwort im Klartext abfragen oder aber den Eintrag bearbeiten sowie löschen. Möchte der Benutzer ein neues Passwort für einen Eintrag vergeben, so kann dies entweder von Hand getan werden oder der Benutzer verwendet den Passwortgenerator von Morik, wodurch zufällige Passwörter variabler Länge generiert werden. Dies steigert die Sicherheit der Passwörter im Vergleich zur manuellen Eingabe.

In den folgenden Kapiteln wird auf die einzelnen Kriterien des Programmentwurfs mit Bezug zum Code von Morik eingegangen.

\chapter{Clean Architecture}
\section{Plugins}


% ---- Literaturverzeichnis
\cleardoublepage
\renewcommand*{\chapterpagestyle}{plain}
\pagestyle{plain}
\pagenumbering{Roman}                   % Römische Seitenzahlen
\setcounter{page}{\numexpr\value{savepage}+1}
\printbibliography[title=Literaturverzeichnis]

% ---- Anhang
\appendix
%\clearpage
%\pagenumbering{Roman}  % römische Seitenzahlen für Anhang

\newpage
\end{document}
